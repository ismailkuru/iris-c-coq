\section{Program Logic}
\label{sec:program-logic}

This section describes how to build a program logic for the C language (\cf \Sref{sec:language}) on top of Iris..

\subsection{Ghost Resource}

Using the standard Iris assertions and the ownership of ghost heap resource, we can define some basic custom assertions:

\begin{align*}
l \mapsto_q v : t \eqdef{}&  l \mapsto_q \encode(v) \land \tychk{v}{t} \\
l \mapsto v : t              \eqdef{}& l \mapsto_1 v : t\\
l \mapsto_q - : t            \eqdef{}& \Exists v. l \mapsto_q v : t\\
f \tmapsto F \eqdef{}& \ownGhost{\Text}{\authfrag \mapsingleton{f}{\aginj F}}
\end{align*}

\subsection{Weakest Precondition}

\paragraph{Defining weakest precondition.}

The shared state interpreting predicate $S : \State \to \iProp$ for our \irisc{} language is defined below,
with threads local state $s$ set to $\Stack \times \Env$ (and initialized to $(\emptyset, \emptyset)$),
as required for instantiating the parametric WP defined for a general class of \texttt{language}.

\[S(\sigma) \eqdef 
  \ownGhost{\Heap}{\authfull \fmap (\lambda v. (1, \aginj v), \sigma.\text{heap})} *
  \ownGhost{\Text}{\authfull \fmap (\aginj, \sigma.\text{text})}
\]

\begin{align*}
  \textdom{wp} \eqdef{}& \MU \textdom{wp}. \Lam \mask, (e, s), \pred. \\
        & (\Exists v. \toval(e) = v \land \pvs[\mask] \pred(v)) \lor {}\\
        & \Bigl(\toval(e) = \bot \,\land \\
        &\qquad \All \state. S(\state) \vsW[\mask][\emptyset] {}\\
        &\qquad \red(e, s, \state) * \later\All e', s', \state', \vec e.
            \cstepf{e}{s}{\state}{e'}{s'}{\state'}{\vec e} \vsW[\emptyset][\mask] {}\\
            &\qquad\qquad S(\state') * \textdom{wp}(\mask, (e', s'), \pred) *
            \Sep_{\expr'' \in \vec e} \textdom{wp}(\top, (\expr'', \mathrm{init}_s), \Lam \any. \TRUE)\Bigr) \\
\end{align*}

Here are some conventions:

\begin{itemize}
\item If we leave away the mask $\mask$, we assume it to be $\top$.
\item $\pred$ in post-condition might or might not take a value parameter, depending on the context.
\item When we elide stack $s$ in WP-rules, like $\wpre{e}{\pred}$, then the rules holds for all such
      WP with any same $s$. This is useful in making local WP rules look less cluttered.
\end{itemize}

\paragraph{Laws of weakest precondition.}
The rules in figure \ref{fig:wp-general} and figure \ref{fig:wp-irisc} can all be derived.

\begin{figure}[!ht]
\label{fig:wp-general}
\begin{mathpar}
\infer[wp-value]
{}{\pred(\val) \proves \wpre{\val, s}[\mask]{\pred}}

\infer[wp-atomic]
{ \textlog{atomic}(e)}
{ \pvs[\mask_1][\mask_2] \wpre{\expr, s}[\mask_2]{\Ret\var. \pvs[\mask_2][\mask_1]\prop}
   \proves \wpre{\expr, s}[\mask_1]{\Ret\var.\prop} }

\infer[wp-strong-mono]
{\mask_1 \subseteq \mask_2}
{(\All v. \Phi(v) \pvs[\mask_2] \Psi(v)) * \wpre{e}[\mask_1]{\pred}
 \proves \wpre{e}[\mask_2]{\Psi}}

\infer[fupd-wp]
{}{\pvs[\mask] \wpre{e}[\mask]{\pred} \proves \wpre{e}[\mask]{\pred}}

\infer[wp-fupd]
{}{\wpre{e}[\mask]{\Ret\var.\pvs[\mask] \pred(\var)} \proves \wpre{e}[\mask]{\pred}}

\end{mathpar}
\caption{General WP rules}
\end{figure}

\begin{figure}[!ht]
\label{fig:wp-irisc}
\begin{mathpar}

\infer[wp-skip]
{}{\later \wpre{e, s}[\mask]{\pred} \proves \wpre{\val; e, s}[\mask]{\pred}}

 \infer[wp-assign]
{\tychk{v}{\tau'} \quad \assigncomp{\tau}{\tau'}}
{ \later l \mapsto - : \tau * \later (l \mapsto v : \tau \wand \pred)
 \proves
 \wpre{l \la v, s}[\mask]{\pred}}

\infer[wp-ret]
{}{\wpre{k'(v), (KS, \Omega_1)}[\mask]{\pred} \proves \wpre{k(\erete{v}), (\kcall{k'}{\Omega_1}::KS, \Omega_2)}[\mask]{\pred}}

\infer[wp-bind]
{\isjmp(e) = \FALSE}
{\wpre{\expr, s}[\mask]{\Ret\var. \wpre{k(\var), s}[\mask]{\pred}}
 \proves 
 \wpre{k(\expr), s}[\mask]{\pred}}

 \infer[wp-op]
{\llbracket oplus \rrbracket (v_1, v_2) = v'}
{\pred(v')
 \proves
 \wpre{v_1 \oplus v_2}[\mask]{\pred}}

 \infer[wp-load]
{}
{ \later l \mapsto_q v : \tau  * \later (l \mapsto_q v : \tau \wand \pred (v))
 \proves
 \wpre{\edereft{\tau}{l}}[\mask]{\pred}}


\end{mathpar}
\caption{WP rules specific to \irisc{}, part I}
\end{figure}


\begin{figure}[!ht]
\label{fig:wp-irisc2}
\begin{mathpar}

\infer[wp-seq]
{\isjmp(e_1) = \FALSE}
{\wpre{e_1, s}[\mask]{v,\,\wpre{v;;e_2, s}[\mask]{\pred}}
 \proves
 \wpre{e_1 ;; e_2, s}[\mask]{\pred}}

\infer[wp-alloc]
{\tychk{v}{\tau}}
{ (\All l. l \mapsto v : \tau \wand \pred(l) ) \proves \wpre{\ealloc{\tau}{v}, s}[\mask]{\pred}}

\infer[wp-if-true]
{}{\later \wpre{e_1, s}[\mask]{\pred}
   \proves \wpre{\eif{\vtrue}{e_1}{e_2}, s}[\mask]{\pred}}

\infer[wp-if-false]
{}{\later \wpre{e_2, s}[\mask]{\pred}
   \proves \wpre{\eif{\vfalse}{e_1}{e_2}, s}[\mask]{\pred}}

\infer[wp-fst]
{}
{\later \pred(v_1) \proves \wpre{\efst{(v_1, v_2)}, s}[\mask]{\pred}}

\infer[wp-snd]
{}
{\later \pred(v_2) \proves \wpre{\efst{(v_1, v_2)}, s}[\mask]{\pred}}

\infer[wp-call]
{}
{ f \tmapsto \Funct(\tau, ps, e) * \later (\wpre{e[ps/ls], (\kcall{k}{\Omega'}::KS, \Omega)}[\mask]{\pred})
  \proves \wpre{k(\ecall{\tau}{f}{ls}), (KS, \Omega')}[\mask]{\pred}}

\infer[wp-cas-fail]
{\tychk{v_1}{\tau} \quad \tychk{v_2}{\tau} \quad v_1 \neq v}
{\later l \mapsto_q v : \tau * \later (l \mapsto_q v : \tau \wand \pred(\vfalse))
 \proves \wpre{\eCAS{\tau}{l}{v_1}{v_2}, s}[\mask]{\pred}}

\infer[wp-cas-suc]
{\tychk{v_1}{\tau} \quad \tychk{v_2}{\tau}}
{\later l \mapsto_q v_1 : \tau * \later (l \mapsto_q v_2 : \tau \wand \pred(\vfalse))
 \proves \wpre{\eCAS{\tau}{l}{v_1}{v_2}, s}[\mask]{\pred}}

\infer[wp-fork]
{}{f \tmapsto \Funct(\tau, \emptyset, e) * \later \pred(\void) * \later (\wpre{e, (\emptyset, \emptyset)}{\Ret\any.\TRUE})
   \proves \wpre{\efork{\tau}{f}{vs}, s}[\mask]{\pred}}

\infer[wp-while]
{}{ \later \wpre{\eif{c}{e;\econtinue}{\ebreak}, (\kwhile{c}{e}{k}::KS, \Omega)}[\mask]{\pred}
    \proves \wpre{ k(\ewhile{c}{e}, (KS, \Omega)}[\mask]{\pred} }

\infer[wp-break]
{}{ \wpre{k'(\void), (KS, \Omega)}[\mask]{\pred}
    \proves \wpre{k(\ebreak), (\kwhile{c}{e}{k'}::KS, \Omega)}[\mask]{\pred} }

\infer[wp-continue]
{}{ \wpre{k'(\ewhile{c}{e}), (KS, \Omega)}[\mask]{\pred}
    \proves \wpre{k(\econtinue), (\kwhile{c}{e}{k'}::KS, \Omega)}[\mask]{\pred} }

\end{mathpar}
\caption{WP rules specific to \irisc{}, part II}
\end{figure}

\subsection{Soundness}

The soundness of WP-style program is proven by showing that it is \emph{adequate}:

\begin{lemma}
For all $e, s, \sigma, \pred: \Val \ra \Prop$,
\begin{align*}
&\TRUE \proves \wpre{e}[\top]{\pred} \Ra \\
&\quad (\All v, s', \sigma'. \cstep{e}{s, \sigma}{v}{s', \sigma'} \Ra \pred(v)) \land \\
&\quad (\All e', \sigma'. \cstep{e}{s, \sigma}{e'}{s', \sigma'} \Ra (\Exists v. e' = v) \lor \red(e', s', \sigma'))
\end{align*}
\end{lemma}

%%% Local Variables:
%%% mode: latex
%%% TeX-master: "iris"
%%% End:

\section{Refinement}\label{sec:refinement}

\subsection{Spec State and Spec Code}

Considering a STS composed of spec code $\scode$ (as below),
spec state $\state : [X \gmapsto \val]$,
and semantic rules $\sstep{\scode}{\sstate}{\scode'}{\sstate'}$
(as well as $\sstepstar{\scode}{\sstate}{\scode'}{\sstate'}$ trivially derived).

\begin{align*}
    \scode : \SpecCode \bnfdef{}&
        \texttt{done}(v?) \mid \texttt{rel}(r: \sstate \ra v? \ra \sstate \ra \Prop) \mid ...
\end{align*}

\subsection{Refinement RA}

We defined a new RA $\refineM$ to capture the history of spec code execution:
\begin{align*}
\view &\eqdef \vmaster \mid \vsnap \\
\refineM &\eqdef \view \times [\SpecState \times \SpecCode]
\end{align*}

The validity of $\refineM$:
\begin{mathpar}
\infer{}{\valid{\refineinj{v}{\emptyset}}}

\infer{}{\valid{\refineinj{v}{[(\sstate, \scode)]}}}

\infer{
\sstep{\scode}{\sstate}{\scode'}{\sstate'}\qquad
\valid{\refineinj{v}{[(\sstate, \scode) :: cs]}}
}
{\valid{\refineinj{v}{[(\sstate', \scode') :: (\sstate, \scode) :: cs]}}}
\end{mathpar}

The multiplication of $\view$ and $\refineM$ is defined as:

\begin{align*}
\vmaster \cdot \any &\eqdef \vmaster\\
\any \cdot \vmaster &\eqdef \vmaster\\
\vsnap \cdot \vsnap &\eqdef \vsnap
\end{align*}

\[
\refineinj{v_1}{cs_1} \cdot \refineinj{v_2}{cs_2} \eqdef
\begin{cases}
\refineinj{v_1 \cdot v_2}{cs_1} \qquad |cs_1| \geq |cs_2|\\
\refineinj{v_1 \cdot v_2}{cs_2} \qquad |cs_1| < |cs_2|
\end{cases}
\]

The disjointness of multiplication is refined as:

\begin{mathpar}
\infer{\Exists cs'. cs_1 \append cs = cs_2 \lor \Exists cs'. cs_2 \append cs = cs_1}
{\refineinj{\vsnap}{cs_1} \disj \refineinj{\vsnap}{cs_2}}

\infer{}{\refineinj{\vsnap}{cs_1} \disj \refineinj{\vmaster}{cs_1 \append cs_1}}

\infer{}{\refineinj{\vmaster}{cs_1 \append cs_1} \disj \refineinj{\vsnap}{cs_1}}
\end{mathpar}

In the end, $\refineM$ can be proven to be a CMRA with an unit element $\refineinj{\vsnap}{\emptyset}$.

\subsection{Refinement Ghost State, Invariant and Rules}

The refinement ghost state \texttt{refineG} contains three part:
one for $refineM$, and two for \emph{paired} ownership of $\SpecCode$ and $\SpecState$.
The predicates are defined as below:

\begin{align*}
\Sstate(\sstate) \eqdef&\, \ownGhost{\SpecState}{(\frac12, \aginj \sstate)} \\
\Scode(\scode) \eqdef&\, \ownGhost{\SpecCode}{(\frac12, \aginj \scode)} \\
\master'(cs) \eqdef&\, \ownGhost{\refineM}{\refineinj{\vmaster}{cs}} \\
\master(c) \eqdef&\, \Exists cs. \ownGhost{\refineM}{\refineinj{\vmaster}{c::cs}}\\
\snap'(cs) \eqdef&\, \ownGhost{\refineM}{\refineinj{\vsnap}{cs}} \\
\snap(c) \eqdef&\, \Exists cs. \ownGhost{\refineM}{\refineinj{\vsnap}{c::cs}}
\end{align*}

Then we define refinement invariant:
\[I_\refineM \eqdef \Exists \sstate, \scode. \Sstate(\sstate) * \Scode(\scode) * \master(\sstate, \scode)\]

With this, we can give refinement style proofs for certain kernel APIs,
using the following derived rules:

\begin{mathpar}
\infer[spec-update]
{\sstep{\scode}{\sstate}{\scode'}{\sstate'}}
{\knowInv{\iname}{I_\refineM} * \Sstate(\sstate) * \Scode(\scode)
 \proves \upd \knowInv{\iname}{I_\refineM} * \Sstate(\sstate') *
         \Scode(\scode') * \snap(\sstate', \scode')}
\end{mathpar}

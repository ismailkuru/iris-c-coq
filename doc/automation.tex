\section{Automation}
\label{sec:auto}

We also developed some basic tactics for automatically solving the goals,
mostly related to our new language, some enhancing what Iris provides.

\subsection{Tactics for ``Symbolic Execution''}

Similar to what exists in Iris's heap-lang, we also provide convenient tactics
including:

\begin{itemize}
    \item \texttt{wp\_bind <p>}: bind to a head term, or a term containing the head term, with pattern \texttt{p}
    \item \texttt{wp\_assign}: make a step by evaluating assignment's head term $l \la v$
        (it also shifts following statements by applying \texttt{wp\_seq} repetitively beforehand)
    \item \texttt{wp\_load}, \texttt{wp\_op}, \texttt{wp\_fst}, \texttt{wp\_snd} are similar to the one above
    \item \texttt{wp\_skip}: skip over any value before a sequencing operator
    \item \texttt{wp\_run}: keep executing as long as any of the above applies, which feels like doing
      symbolic execution automatically
    \item \texttt{wp\_ret}: return if the head expression is already value
\end{itemize}

\subsection{Tactics for Algebraic Simplification}

\paragraph{\texttt{gmap} RA.}

\texttt{gmap\_simplify}: simplify expressions involving \texttt{gmap} based on some algebraic rules

\paragraph{$\refineM$ RA.}

\texttt{rewrite\_op\_cfgs}: Simplify product of the configuration list component.

\subsection{Misc Tactics}

\paragraph{Inversion. } \texttt{inversion\_estep}, \texttt{inversion\_cstep\_as}, and \texttt{inversion\_jstep\_as}
are all designed to automatically match stepping relation assumption and give proper names to the important results
produced by \texttt{inversion}.

\paragraph{Evaluation Context Equality. } \texttt{gen\_eq <H> <E1> <E2> <KS>} can generate equalities
between expressions \texttt{E1} and \texttt{E2}, and between \texttt{KS} and empty context.
It assumes that both \texttt{E1} and \texttt{E2} are normal form, and there is equality between
the filled ones: \texttt{fill\_ectxs E1 KS = E2}.

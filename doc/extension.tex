\section{Extensions}\label{sec:extension}


Besides the core language and its logic,
we built several extension on top of it in a compatible way.

\subsection{WP-\texttt{ret}}\label{def:wpr}

We extend the unary-exit WP into a binary-exit $\wprer{e}[\mask]{\pred}{\predret}$,
by defining it as below:

\begin{align*}
  \textdom{wp}^{+} \eqdef{}& \MU \textdom{wp}^{+}. \Lam \mask, e, \pred, \predret. \\
        & (\Exists v. \toval(e) = v \land \pred(v)) \lor {}\\
        & (\toval(e) = \bot \,\land \Exists e_h, K. e = K(e_h) \land \Bigl({}\\
        &\qquad (\All s. \isjmp(e_h) = \FALSE * \wpre{e_h; s}[\mask]{\Ret v. \wprer{K(v)}[\mask]{\pred}{\predret}}) \lor \\
        &\qquad (\Exists v. e_h = \erete{v} * \later \predret(v)) \lor \\
        &\qquad (\Exists f, ps, ls, F. \\
                           & \qquad \qquad f \tmapsto \Funct(\any, ps, F) * \\
                           & \qquad \qquad \later (\wprer{F[ps/ls]}[\mask]{\Ret\any.\FALSE}{\Ret v.
                             \wprer{K(v)}[\mask]{\pred}{\predret}})) \\
        &\Bigr) )\\
\end{align*}

And this definition supports the following inference rules:

\begin{mathpar}
\infer[wpr-value]{}{\pred(v) \proves \wprer{v}[\mask]{\pred}{\predret}}

\infer[wpr-ret]{}{\predret(v) \proves \wprer{\erete{v}}[\mask]{\pred}{\predret}}

\infer[wpr-bind]
{}
{\wprer\expr[\mask]{\Ret\var. \wprer{k(\var)}[\mask]{\pred}{\predret}}{\predret}
 \proves 
 \wprer{k(\expr)}[\mask]{\pred}{\predret}}

\infer[wpr-seq]
{}
{\wprer{e_1}{\Ret\var. \wprer{\var ; e_2}[\mask]{\pred}{\predret}}{\predret}
 \proves 
 \wprer{e_1;; e_2}[\mask]{\pred}{\predret}}

\infer[wpr-call]
{}
{ f \tmapsto \Funct(\tau, ps, e) * \later (\wprer{e[ps/ls]}[\mask]{\Ret\any.\FALSE}{\pred})
  \proves \wprer{k(\ecall{\tau}{f}{ls})}[\mask]{\pred}{\predret}}

\infer[wpr-op]
{\llbracket oplus \rrbracket (v_1, v_2) = v'}
{\pred(v')
 \proves
 \wprer{v_1 \oplus v_2}[\mask]{\pred}{\predret}}

\infer[wp-call-r]
{f \tmapsto \Funct(\tau, ps, e) *
 \later \wprer{e[ps/ls]}[\mask]{\Ret\any.\FALSE}{\Ret v. \wpre{k(v); s}[\mask]{\pred}} }
{\wpre{k(\ecall{\tau}{f}{ls}); s}[\mask]{\pred}}

\end{mathpar}

Note how WP-CALL, WP-RET, WP-SEQ, and WP-BIND are simplified in their new, corresponding versions,
also the fact that we can recover local evaluation like WP-OP trivially
(and we won't duplicate too much here).

We also features a rule to compose two styles together freely: WP-CALL-R.

\subsection{Array}

Using the most primitive form of aggregated structure, product type, we can easily derive
array (implemented) or even \texttt{struct} in future work.

``A pointer $p$ pointing to an array $vs$ of $n$ elements which are $\tau$-typed'' is expressed with:
\[p \mapsto_q vs: \tau^n\]

In implementation, $vs$ is \texttt{varray (l:list val)}, and $\tau^n$ is \texttt{tyarray (t:type) (n:nat)}.

We also have an intermediate representation of a slice of array:
\[p \mapsto_q [v_i : \tau, ..., v_{i + l - 1}:\tau]\]

The first form is easier for allocation, since it is defined as a unified aggregate structure;
but the second form is easier for use, for example, splitting or indexing. We have the following lemmas proven in Coq:

\begin{mathpar}
\infer[split-slice]{}
{p \mapsto_q [v_i : \tau, ..., v_{i + l_1 - 1}:\tau] * p \mapsto_q [v_{i + l_1} : \tau, ..., v_{i + l_1 + l_2 - 1}:\tau]
 \equiv p \mapsto_q [v_i : \tau, ..., v_{i + l_1 + l_2 - 1}:\tau]}

\infer[index-spec]{i < n}
{\hoare{p \mapsto_q vs: \tau^n}{\edereft{\tau}{p + i}}
       {\Ret v. p \mapsto_q vs: \tau^n * \lookupmap{vs}{i} = \Some{v}}}

\end{mathpar}

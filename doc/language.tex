\section{Language}
\label{sec:language}

\subsection{Definitions}

\paragraph{Syntax.}\label{p:type}

The language is a simplified version of C. It consists of \Stmts{} (statements), \Prim{} (primitives),
\Expr{} (expressions), \Type{} (types) and \Val{} (values).

Memory address $l: \Addr \eqdef \vaddr{b}{o}$ where block address $b \in \nat$, offset $o \in \integer^{+}$.

\begin{align*}
    \tau : \Type \bnfdef{}&
        \tyvoid \mid \tynull \mid \tybyte \mid \tyword \mid \typtr{\tau} \mid \tau \times \tau
\\
    v : \Val \bnfdef{}&
        \void \mid \vnull \mid i \in [0, 2^8) \mid i \in [0, 2^{32}) \mid l \mid (v, v)
\\
    e : \Expr \bnfdef{}&
        v \mid x \mid e \oplus e \mid \deref{e} \mid \dereft{\tau}{e} \mid \& e \mid \ecast{e}{\tau} \mid \efst{e} \mid \esnd{e}
\\
    p : \Prim \bnfdef{}&
        \pcli \mid \psti
\\
    s : \Stmts \bnfdef{}&
        \sskip \mid p \mid e \la e \mid \sif{e}{s}{s} \mid \swhile{e}{e}{s} \mid \\&
        \sret \mid \srete{e} \mid \scall{f}{e_1, ..., e_n} \mid s ; s
\\
\end{align*}

XXX: Explain two type of dereference.

\paragraph{Program.}

A program is considered to be a set of functions, each identified by its name. Each
function is a triple of return type $\tau_{ret}$, parameter declarations $(x_1 : \tau_1, ...)$,
and function body $s$.

\paragraph{Evaluation Context.}

To make the evaluation order explicit and reusable by the $\text{WP-BIND}$ rule, we define evaluation context.
Compared to a simple expression-based language which has only one kind of context $\ctx$, we define two contexts
$\ctx_e, \ctx_s$ for both $\Expr$ and $\Stmts$:

\begin{align*}
    \ctx_e : \ectx \eqdef{}&
        \emptyctx \mid \emptyctx \oplus e \mid v \oplus \emptyctx \mid
        \deref{\emptyctx} \mid \& \emptyctx \mid \ecast{\emptyctx}{\tau} \mid
        \efst{\emptyctx} \mid \esnd{\emptyctx}
\\
    \ctx_s : \sctx \eqdef{}&
        \emptyctx \la e \mid l \la \emptyctx \mid \sif{\emptyctx}{s}{s} \mid \swhile{e}{\emptyctx}{s} \mid \\&
         \srete{\emptyctx} \mid \scall{f}{v_1, ..., \emptyctx, e_1, ...}
\end{align*}

Now we can define \emph{context}: $\ctx : \allctx \eqdef (\ctx_e, \ctx_s)$.

\subsection{Semantics}
\paragraph{Model.}

Define $c: \Code \eqdef (\cureval: \{ \Expr + \Stmts\}, \curctx: \allctx, \ctx^{*}: [\allctx])$, in which

\begin{itemize}
\item $\cureval$ is the ``current evaluation'', which could be either an expression $e^\dag$ or a statement $s^\dag$.
\item $\curctx$ is the ``current context'', which semantically means rest code to execute in current frame.
\item $\ctx^{*}$ is the ``previous contexts'', which semantically means previous frames on call stack
\end{itemize}

Next, we define byte-size value and memory model:

\[\bval : \byteval \eqdef \mundef \mid i \in [0, 2^8) \mid l_{\{0 \mid 1 \mid 2 \mid 3\}} \mid \vnull\]

We define state $\sigma: \State \eqdef [ l \gmapsto \bval ]$,
i.e. a heap in which every address maps to a byte-level value.

You may notice the difference between $\Val$ and $\byteval$, it is intentional to create a layer of abstraction for
easier manipulation of spatial assertions and also a clean, unified language syntax.

\paragraph{Small-Step Operational Semantics.}

We define the HNF small step semantics for both expression ($\estep{e}{e'}{\sigma}$) and statements ($\sstep{s}{\sigma}{s'}{\sigma'}$),
and then combine them together ($\cstep{e^\dag}{\sigma}{e'^\dag}{\sigma}$, $\cstep{s^\dag}{\sigma}{s'^\dag}{\sigma'}$).

\begin{figure}
\begin{mathpar}

\infer[ES-BINOP]{\llbracket oplus \rrbracket (v_1, v_2) = v'}{\estep{v_1 \oplus v_2}{v'}{\sigma}}

\infer[ES-DEREF]{\tychk{v}{t} \quad \sigma \vdash l \mapsto \encode(v)}{\estep{\dereft{\tau}{l}}{v}{\sigma}}

\infer[ES-FST]{}{\estep{\efst{(v_1, v_2)}}{v_1}{\sigma}}

\infer[ES-SND]{}{\estep{\esnd{(v_1, v_2)}}{v_2}{\sigma}}

\end{mathpar}
\caption{Semantics of expression evaluation}
\end{figure}

\begin{figure}
\begin{mathpar}

\infer[SS-ASSIGN]{}{\sstep{l \la v}{\sigma}{\sskip}{\sigma[ l \mapsto \encode(v) ]}}

\infer[SS-SEQ]{}{\sstep{\sskip ; s}{\sigma}{s}{\sigma}}

\infer[SS-SEQ-HEAD]{\sstep{s_1}{\sigma}{s_1'}{\sigma'}}{\sstep{s_1; s_2}{\sigma}{s_1'; s_2}{\sigma'}}

\end{mathpar}
\caption{Semantics of statements execution}
\end{figure}



XXX: Define $\sigma \vdash l \mapsto \vec \bval$

\subsection{Type System and Environment}

\paragraph{Local Typing Rules.}

The types are defined in \Sref{p:type}, and all values in $\val$ can be \emph{locally} typed trivially (since $\val$ is introduced to reflect type structure in some sense). Nevertheless, due to the fact that language of study is weakly-typed in the vein of C, we still have some ``weird'' rules worth documenting.

Here we define local typing judgment $\tychk{v}{\tau}$ for values.

\begin{mathpar}
\infer[tychk-void]{}{\tychk{\void}{\tyvoid}}

\infer[tychk-null]{}{\tychk{\vnull}{\tynull}}

\infer[tychk-int8]{}{\tychk{i \in [0, 2^8)}{\tybyte}}

\infer[tychk-int32]{}{\tychk{i \in [0, 2^{32})}{\tyword}}

\infer[tychk-null-ptr]{}{\All \tau. \tychk{\vnull}{\typtr{\tau}}}

\infer[tychk-ptr]{}{\All \tau, l. \tychk{l}{\typtr{\tau}}}

\infer[tychk-prod]{\tychk{v_1}{\tau_1} \quad \tychk{v_2}{\tau_2}}{\tychk{(v_1, v_2)}{\tau_1 \times \tau_2}}

\end{mathpar}

\paragraph{Typing Environment}

When variables are introduced, we will need an environment $\Gamma : \Env \eqdef [x \gmapsto (\tau, l)]$
to ``unfold'' the meaning of variables.

In Iris-OS, variables are all unfolded before the running the function body,
which saves the program logic from caring about the lexical environment.
During this unfolding, we will replace variables with either their location (left-hand side)
or the dereference of their location (right-hand side). And we will also produce a pointer arithmetic
expression when processing the left-hand side expression, which requires type inference $\tyinf{e}{\tau}$
(since it is standard and trivial, we will leave out the details here).

Then we define the rules for ``interpreting'' left-hand side expression ($\reslhs{e}$ or $\reslhs{s}$)
and right-hand side expression ($\resrhs{e}$ or $\resrhs{s}$). In all rules below, a $\Gamma$ is implicitly
captured, and if any operation on $\Gamma$ failed, then the rule will be returning ``invalid'' in implementation.

\begin{align*}
\reslhs{e_1 \la e_2} =&\, \reslhs{e_1} \la e_2\\
\reslhs{x} =&\, \Gamma(x).l \\
\reslhs{\deref{e}} =&\, \resrhs{e} \\
\reslhs{\efst{e}} =&\, \reslhs{e} \\
\reslhs{\esnd{e}} =&\, \reslhs{e} + \sizeof(\tau) \quad \text{if } \tyinf{\reslhs{e}}{\typtr{(\tau_1 \times \tau_2)}}\\
\reslhs{l} =&\, l\\
\\
\resrhs{e_1 \la e_2} =&\, e_1 \la \resrhs{e_2}\\
\resrhs{x} =&\, \dereft{\Gamma(x).t}{\Gamma(x).l} \\
\resrhs{\deref{e}} =&\, \dereft{\tau}{\resrhs{e}} \quad \text{if } \tyinf{e}{\typtr{\tau}}
\end{align*}

The cases not covered are defined recursively (and trivially)

%%% Local Variables:
%%% mode: latex
%%% TeX-master: "iris"
%%% End:

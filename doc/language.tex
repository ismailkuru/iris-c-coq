\section{Language}
\label{sec:language}

\subsection{Definitions}

\paragraph{Syntax.}\label{p:type}

The language is a simplified version of C. It consists of \Stmts{} (statements), \Prim{} (primitives),
\Expr{} (expressions), \Type{} (types) and \Val{} (values).

Memory address $l: \Addr \eqdef \vaddr{b}{o}$ where block address $b \in \nat$, offset $o \in \integer^{+}$.

\begin{align*}
    \tau : \Type \bnfdef{}&
        \tyvoid \mid \tynull \mid \tybyte \mid \tyword \mid \typtr{\tau} \mid \tau \times \tau
\\
    v : \Val \bnfdef{}&
        \void \mid \vnull \mid i \in [0, 2^8) \mid i \in [0, 2^{32}) \mid l \mid (v, v)
\\
    e : \Expr \bnfdef{}&
        v \mid x \mid e \oplus e \mid \deref{e} \mid \& e \mid \ecast{e}{\tau} \mid \efst{e} \mid \esnd{e}
\\
    p : \Prim \bnfdef{}&
        \pcli \mid \psti
\\
    s : \Stmts \bnfdef{}&
        \sskip \mid p \mid e \la e \mid \sif{e}{s}{s} \mid \swhile{e}{s} \mid \\&
        \sret \mid \srete{e} \mid \scall{f}{e_1, ..., e_n} \mid s ; s
\\
\end{align*}

\paragraph{Program.}

A program is considered to be a set of functions, each identified by its name. Each
function is a triple of return type $\tau_{ret}$, parameter declarations $(x_1 : \tau_1, ...)$,
and function body $s$.

\paragraph{Evaluation Context.}

To make the evaluation order explicit and reusable by the $\text{WP-BIND}$ rule, we define evaluation context.
Compared to a simple expression-based language which has only one kind of context $\ctx$, we define two contexts
$\ctx_e, \ctx_s$ for both $\Expr$ and $\Stmts$:

\begin{align*}
    \ctx_e : \ectx \eqdef{}&
        \emptyctx \mid \emptyctx \oplus e \mid v \oplus \emptyctx \mid
        \deref{\emptyctx} \mid \& \emptyctx \mid \ecast{\emptyctx}{\tau} \mid
        \efst{\emptyctx} \mid \esnd{\emptyctx}
\\
    \ctx_s : \sctx \eqdef{}&
        \emptyctx \la e \mid l \la \emptyctx \mid \sif{\emptyctx}{s}{s} \mid \swhile{\emptyctx}{s} \mid \\&
         \srete{\emptyctx} \mid \scall{f}{v_1, ..., \emptyctx, e_1, ...}
\end{align*}

Now we can define \emph{context}: $\ctx : \allctx \eqdef (\ctx_e, \ctx_s)$.

\subsection{Semantics}
\paragraph{Model.}

Define $c: \Code \eqdef (\cureval: \{ \Expr + \Stmts\}, \curctx: \allctx, \ctx^{*}: [\allctx])$, in which

\begin{itemize}
\item $\cureval$ is the ``current evaluation'', which could be either an expression $e^\dag$ or a statement $s^\dag$.
\item $\curctx$ is the ``current context'', which semantically means rest code to execute in current frame.
\item $\ctx^{*}$ is the ``previous contexts'', which semantically means previous frames on call stack
\end{itemize}

Next, we define byte-size value and memory model:

\[\bval : \byteval \eqdef \mundef \mid i \in [0, 2^8) \mid l_{\{0 \mid 1 \mid 2 \mid 3\}} \mid \vnull\]

We define state $\sigma: \State \eqdef [ b \gmapsto [\bval] ]$,
i.e. a heap in which every block maps to a list of byte-size values.

You may notice the difference between $\Val$ and $\byteval$, it is intentional to create a layer of abstraction for
easier manipulation of spatial assertions and also a clean, unified language syntax.

\paragraph{Small-Step Operational Semantics.}

TODO.

\subsection{Type System and Environment}

\paragraph{Typing Rules.}

The types are defined in \Sref{p:type}, and all values in $\val$ can be \emph{locally} typed trivially (since $\val$ is introduced to reflect type structure in some sense). Nevertheless, due to the fact that language of study is weakly-typed in the vein of C, we still have some ``weird'' rules worth documenting.

First, we define local typing judgement $\vdash v : \tau$ for values.

\begin{mathpar}
\infer[tychk-void]{}{\vdash \void : \tyvoid}

\infer[tychk-null]{}{\vdash \vnull : \tynull}

\infer[tychk-int8]{}{\vdash i \in [0, 2^8) : \tybyte}

\infer[tychk-int32]{}{\vdash i \in [0, 2^{32}) : \tyword}

\infer[tychk-int8-to-32]{\vdash v : \tybyte}{\vdash v : \tyword}

\infer[tychk-int32-to-8]{\vdash i \in [0, 2^8) : \tyword}{\vdash i : \tybyte}

\infer[tychk-null-ptr]{}{\All \tau. \vdash \vnull : \typtr{\tau}}

\infer[tychk-ptr]{}{\All \tau, l. \vdash l : \typtr{\tau}}

\infer[tychk-ptr-cast]{\vdash v : \typtr{\tau}}{\vdash v : \typtr{\tau'}}

\infer[tychk-prod]{\vdash v_1 : \tau_1, v_2 : \tau_2 }{\vdash (v_1, v_2) : \tau_1 \times \tau_2}

\end{mathpar}

%%% Local Variables:
%%% mode: latex
%%% TeX-master: "iris"
%%% End:
